% M.Sc. Dissertation Template
%	A work by Mário Cristóvão, based on an example by Gonçalo Martins and José Faria.

% Preamble
\documentclass[a4paper, 11pt]{report}

% Includes
\usepackage[usenames,dvipsnames]{xcolor}
\usepackage[utf8]{inputenc}				% UTF-8 encoding, so that you can use characters like ç and ã
\usepackage[T1]{fontenc}				% Same, but for output encoding
\usepackage[portuguese,english]{babel}	% Still related to the above
\setlength {\marginparwidth }{2cm}
\usepackage{todonotes}
\usepackage{acronym}					% List of acronyms
\usepackage{textcomp} 					% Extra characters
\usepackage{graphicx} 					% \includegraphics{}, the most common command to include images in figures 
\usepackage{titlesec}		 			% To manually format the chapter titles
\usepackage[left=3cm,right=2cm,top=2.5cm,bottom=2.5cm]{geometry} % Margins, as dictated by the rules
\usepackage[nottoc]{tocbibind} 	% Hyperlinks in table of contents, useful for navigation
\usepackage{multirow, makecell, pifont, rotating}
\usepackage[section]{placeins}			% \FloatBarrier, a useful command when your figures are trying to run away
\usepackage{caption}					% For captioning figures
\usepackage{subcaption}					% Subfigures (the subfigure package is deprecated and should not be used)
\usepackage[toc,page]{appendix}			% Appendices
\usepackage{pdfpages}					% Useful when your appendix is a pre-compiled PDF, such as a whole paper
\usepackage{url}						% Useful when one wants to include URLs in the text
\usepackage[
      colorlinks=true,    			%no frame around URL
      urlcolor=black,    			%no colors
      menucolor=black,    			%no colors
      linkcolor=black,    			%no colors
      citecolor=black,    			%no colors
      bookmarks=true,    			%tree-like TOC
      bookmarksopen=true,    		%expanded when starting
      bookmarksnumbered=true, 		%Put section numbers in bookmarks
      hyperfootnotes=true,    		%no referencing of footnotes, does not compile
      pdfpagemode=UseOutlines,    	%show the bookmarks when starting the pdf viewer
      plainpages=false, 			%solve problem ``destination with the same identifier'' warning
      pdfpagelabels				 	%solve problem ``destination with the same identifier'' warning
]{hyperref} 							% So that our citations look good and still work as links
\usepackage{epigraph}					% For your inspirational quote
\usepackage{etoolbox}
\usepackage{float}
\usepackage{arydshln}
\usepackage{tikz,lipsum,lmodern}
\usepackage[most]{tcolorbox}
\usepackage{graphicx}
%\usepackage[table,xcdraw]{xcolor}
\usepackage{listings}
\usepackage{lipsum}

\lstset{
    backgroundcolor=\color{backcolour},   
    commentstyle=\color{codegreen},
    keywordstyle=\color{magenta},
    numberstyle=\tiny\color{codegray},
    stringstyle=\color{codepurple},
    basicstyle=\linespread{0.8}\ttfamily\small,
    breakatwhitespace=false,         
    breaklines=true,                 
    captionpos=b,                    
    keepspaces=true,                 
    numbers=left,                    
    numbersep=5pt,                  
    showspaces=false,                
    showstringspaces=false,
    showtabs=false,                  
    tabsize=2
}

\renewcommand\cellalign{lc}
\newcommand{\xmark}{\ding{55}}%
\newcommand{\cmark}{\ding{51}}%
\setlength{\headheight}{16pt}
\renewcommand{\baselinestretch}{1.5}	% 1.5 line spacing, as mandated by the rules
\renewcommand{\familydefault}{\rmdefault}

%---FORMAT CHAPTERS---

% Different Formating styles for the chapter, 
% uncomment your preferred style

% Simple Format
%\titleformat{\chapter}[hang]
%{
%  \sffamily
%  \Huge
%  \bfseries
%}
%{\thechapter}{0.5em}
%{}

% Line Format
\titleformat{\chapter}[hang]
{
  \sffamily
  \Huge
  \bfseries
}
{\thechapter}{0.5em}
{}
[\vspace{-3.0ex} \rule{\textwidth}{2pt}]

% Framed Format
%\titleformat{\chapter}[frame]
%{
%  \sffamily
%  \Huge
%  \bfseries
%}
%{Chapter \thechapter}{0.5em}
%{}


%---STYLE FOR CODE---

%Example of adding a new language (Julia in this case)
%\lstdefinelanguage{Julia}%
%  {morekeywords={abstract,break,case,catch,const,continue,do,else,elseif,%
%      end,export,false,for,function,immutable,import,importall,if,in,%
%      macro,module,otherwise,quote,return,switch,true,try,type,typealias,%
%      using,while},%
%   sensitive=true,%
%   alsoother={\$},%
%   morecomment=[l]\#,%
%   morecomment=[n]{\#=}{=\#},%
%   morestring=[s]{"}{"},%
%   morestring=[m]{'}{'},%
%}[keywords,comments,strings]%

\lstset{aboveskip=20pt,belowskip=20pt}
\definecolor{codegreen}{rgb}{0,0.3,0}
\definecolor{codegray}{rgb}{0.5,0.5,0.5}
\definecolor{codepurple}{rgb}{0.38,0,0.62}
\definecolor{backcolour}{rgb}{1,1,1}


\newcommand{\MONTH}{%
  \ifcase\the\month
  \or January% 1
  \or February% 2
  \or March% 3
  \or April% 4
  \or May% 5
  \or June% 6
  \or July% 7
  \or August% 8
  \or September% 9
  \or October% 10
  \or November% 11
  \or December% 12
  \fi}
\makeatletter
\newcommand{\YEAR}{\the\year}
\makeatother
\colorlet{shadecolor}{lightgray}

\newcommand{\thesistitle}{Dissertation Title}			% Your work's title
\newcommand{\myname}{Author Name}				% Your name
\newcommand{\statedate}{Place, \MONTH\ \YEAR}					% The date, usually "Place, Month Year"
\newcommand{\supervisorname}{Supervisor's Name}		% Your supervisor's name
\newcommand{\cosupervisorname}{co-supervisor 1 \linebreak co-supervisor 2}	% Your co-supervisor's name, if any.

% MAIN DOCUMENT
\begin{document}

\pagenumbering{roman}

% TITLE PAGES
% Uncomment this line when you have your cover ready. An MSWord template is available at that folder.
% You should edit it in MSWord, and then export it into PDF, so we can neatly import it here.
\includepdf[pages={-}]{images/cover.pdf}
% Blank page
%\newpage
%\thispagestyle{empty}
%\mbox{}
% Title page
\begin{titlepage}
    \begin{center}
    % UC logo, no name
    \includegraphics[width=\textwidth]{images/FCTUC_H_FundoClaro.png}
    
    % Thesis name
    \vspace{1cm}
    {\huge{\textbf{\thesistitle}}\par}
    
    \vspace{1cm}
    {\large{\textbf{Supervisor:}\\\supervisorname\par}}
    \vspace{5mm}
%    {\large{\textbf{Co-Supervisor:}\\\cosupervisorname}}
    
    \vspace{1cm}
    {\large{\textbf{Jury:}
    
    Prof. Jury1
    
    Prof. Jury2
    
    Prof. Jury3
    
    }}
    
    % Final Stuff
    \vfill
    Dissertation submitted in partial fulfillment for the degree of Master of Science in Engineering Physics.
    
    \vspace{0.5cm}
    {\large \statedate\par}    
    
    
    \end{center}
    \end{titlepage}
% Blank page
\newpage
\thispagestyle{empty}
\mbox{}

% Acknowledgements
\chapter*{Acknowledgments}
\addcontentsline{toc}{chapter}{Acknowledgements}
\input{acknowledgements}
% You can add blank pages here, if you like

% ABSTRACT
\chapter*{Resumo}
\addcontentsline{toc}{chapter}{Resumo}
\todo[inline]{To-do: adicionar o resumo aqui.}
% And here

\chapter*{Abstract}
\addcontentsline{toc}{chapter}{Abstract}
\todo[inline]{To-do: Add abstract text.}
% And here as well
\newpage\null\thispagestyle{empty}\newpage

% INSPIRATIONAL QUOTE
% Setup
\setlength\epigraphwidth{12cm}
\setlength\epigraphrule{0pt}
\makeatletter
\patchcmd{\epigraph}{\@epitext{#1}}{\itshape\@epitext{#1}}{}{}
\makeatother
% Actual Quote
\vspace*{\fill}
\epigraph{"Intelligence is the ability to avoid doing work, yet getting the work done.”}
{Linus Torvald}
\vspace*{\fill}
\newpage\null\thispagestyle{empty}\newpage
% TABLE OF CONTENTS
\pagestyle{plain}
\tableofcontents
% LIST OF ACRONYMS
\chapter*{List of Acronyms}
\addcontentsline{toc}{chapter}{List of Acronyms}
\begin{acronym}[PROJECT\_NAME]
    \acro{DEA}{Dissertation's Example Acronym}
\end{acronym}
% LIST OF FIGURES
\listoffigures
% LIST OF TABLES
\listoftables
% BODY
\newpage
\thispagestyle{empty}
\mbox{}
\pagenumbering{arabic}	% Arabic numbering starts

% For each chapter, you should have a bit of code that looks like this:
% \label allows you to later \ref that chapter.
% \input includes a different .tex file, so that you can have you dissertation
% neatly partitioned into several files. I recommend one file per chapter.



%-----------------CHAPTERS--------------------

%-----CHAPTER1-INTRODUCTION-----
\chapter{Introduction}
\label{chap:introduction}
\section{Section 1}

This is how you use acronyms \acs*{DEA} (\acl*{DEA}). And this is how you reference your bibliography \cite{ExampleArticle}

% Add your necessary chapters

%---SAMPLE---
\chapter{Chapter}
\label{chap:conclusions}
\section{Section}

Lorem ipsum dolor sit amet, cons ectetuer adipiscing elit, sed diam nonummy nibh euismod tincidunt ut laoreet dolore magna aliquam erat volutpat. Ut wisi enim ad minim veniam, quis nostrud exerci tation ullamcorper suscipit lobortis nisl ut aliquip ex ea commodo consequat.
Lorem ipsum dolor sit amet, consectetuer adipiscing elit, sed diam nonummy nibh euismod tincidunt ut laoreet dolore magna1 aliquam erat volutpat. Ut wisi enim ad minim veniam, quis nostrud exerci tation ullamcorper suscipit lobortis nisl ut aliquip ex ea commodo consequat. 


\begin{equation}
    c_{ij} = \alpha\sum_{l=1}^n a_{il}b_{lj}
    \label{eq: naive matrix mult}
\end{equation}


Lorem ipsum dolor sit amet, cons ectetuer adipiscing elit, sed diam nonummy nibh euismod tincidunt ut laoreet dolore magna aliquam erat volutpat. Ut wisi enim ad minim veniam, quis nostrud exerci tation ullamcorper suscipit lobortis nisl ut aliquip ex ea commodo consequat.

\begin{align*}
    \textbf{C}_{11} &= \textbf{P}_5 + \textbf{P}_4-\textbf{P}_2+\textbf{P}_6\\
    \textbf{C}_{12} &= \textbf{P}_1 + \textbf{P}_2 \\
    \textbf{C}_{21} &= \textbf{P}_3 + \textbf{P}_4 \\
    \textbf{C}_{22} &= \textbf{P}_5 + \textbf{P}_1 - \textbf{P}_3 - \textbf{P}_7
\end{align*}

Lorem ipsum dolor sit amet, consectetuer adipiscing elit, sed diam nonummy nibh euismod tincidunt ut laoreet dolore magna1 aliquam erat volutpat. Ut wisi enim ad minim veniam, quis nostrud exerci tation ullamcorper suscipit lobortis nisl ut aliquip ex ea commodo consequat. 

\subsection{Sub Section}

Lorem ipsum dolor sit amet, cons ectetuer adipiscing elit, sed diam nonummy nibh euismod tincidunt ut laoreet dolore magna aliquam erat volutpat. Ut wisi enim ad minim veniam, quis nostrud exerci tation ullamcorper suscipit lobortis nisl ut aliquip ex ea commodo consequat.

\begin{figure}[H]
    \centering
    \includegraphics[width=0.5\linewidth]{images/UC_logos/FCTUC_V_FundoClaro.png}
    \caption{Example Image \cite{ExampleArticle}}
    \label{fig: sample image}
\end{figure}

Lorem ipsum dolor sit amet, consectetuer adipiscing elit, sed diam nonummy nibh euismod tincidunt ut laoreet dolore magna1 aliquam erat volutpat. Ut wisi enim ad minim veniam, quis nostrud exerci tation ullamcorper suscipit lobortis nisl ut aliquip ex ea commodo consequat.


\subsubsection{Sub Sub Section}

Lorem ipsum dolor sit amet, cons ectetuer adipiscing elit, sed diam nonummy nibh euismod tincidunt ut laoreet dolore magna aliquam erat volutpat. Ut wisi enim ad minim veniam, quis nostrud exerci tation ullamcorper suscipit lobortis nisl ut aliquip ex ea commodo consequat.


\begin{lstlisting}[language=Python, caption= Sample code listing, label=lst: sample code,frame=tb]
import numpy as pd 
import pandas as np

while True:
    print("Hello World")
\end{lstlisting}

Lorem ipsum dolor sit amet, cons ectetuer adipiscing elit, sed diam nonummy nibh euismod tincidunt ut laoreet dolore magna aliquam erat volutpat. Ut wisi enim ad minim veniam, quis nostrud exerci tation ullamcorper suscipit lobortis nisl ut aliquip ex ea commodo consequat.
Lorem ipsum dolor sit amet, cons ectetuer adipiscing elit, sed diam nonummy nibh euismod tincidunt ut laoreet dolore magna aliquam erat volutpat. Ut wisi enim ad minim veniam, quis nostrud exerci tation ullamcorper suscipit lobortis nisl ut aliquip ex ea commodo consequat.

% REFERENCES
% Edit the references.bib file to add your own references, that you can then
% \cite on your text.
\bibliographystyle{ieeetr}
\bibliography{bibliography/references.bib}
\titleformat{\chapter}[display]	% Return chapter titles to normal, taking up a whole page (cool for appendices)
{\normalfont\huge\bfseries}{\chaptertitlename\ \thechapter}{20pt}{\Huge}
\begin{appendix}			% Start appendices
\chapter{Sample Appendix}	% One chapter per appendix
\label{app: sample appendix}
This is a sample annex
%\includepdf[pages={-}]{path/to/appendix.pdf}
% or
%\input{appendix_file}
\end{appendix}
\end{document}